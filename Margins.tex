\documentclass[a4paper, 12pt]{report}

\begin{document}

Flow separation is one of the most important things to avoid in the design of aircraft wings. Flow separation is accompanied by dramatic increase in drag and loss of lift. When an airfoil is tested in the transonic regime, the flow over the airfoil goes supersonic on the front of the airfoil. In the pressure recovery phase where the flow slows down to subsonic speeds, a shock wave might occur. This shockwave impinges on the airfoil and typically separation follows at this point. Airfoils which are not designed at transonic speeds would experience poor performance at transonic flight speeds.

\end{document}
