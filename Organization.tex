\documentclass[a4paper, 12pt]{report}

\begin{document}

\part{Engineering Problems: Fluid Mechanics}
We get to choose a subset of a wide variety of engineering problems. One of the hardest engineering problems without a fully definitive answer would be fluid mechanics. Fluid mechanics have been studied for a long time, yet scientists have failed to provide a definitive general analytical solution to fluid mechanics. It might be because of the chaotic nature of the fluid mechanic basic governing equations.

\chapter{Incompressible Flows}
Fluid Flow is often complex. Fluid flow can be thought to be divisible as compressible and incompressible flows. Flows which are low in relative speed can be considered incompressible meanwhile flows which are high in relative speed can be considered compressible. To determine compressibility, one has to look at the Mach number, a non-dimensional parameter that is the ratio of the fluid speed relative to local sound speed.

\section{Inviscid Thin Airfoil Theory}
Engineers seeking to design aircrafts and other wing-like devices must familarize themselves with the basics of inviscid thin airfoil theory. Inviscid thin airfoil theory relies on alot of assumptions and may not be very useful in the most general cases of thicker wings and higher angles of attack, or during stall, but inviscid thin airfoil theory can match experimental data very well as long as its underlying assumptions are not violated.

\subsection{Theoretical and Mathematical Basis}
Thin Airfoil theory relies on alot of mathematical operations that can be considered "expensive". To name a few, one must be familiar with integration. One should also be able to solve a large linear system of equations.

\subsubsection{Integral Operations}
The integrals in thin airfoil theory can be used to determine coefficients of A which can then be used to determine more useful properties like lift coefficient and moment coefficients.

\paragraph{Basic-Pre-Requisite Knowledge}
When approaching thin airfoil theory, one must be prepared to see the knowledge to be a kind of tree. If one is not familar with solving systems of linear equations or the integral operations that often plague thin airfoil theory, one must first familiarize themselves with the "basic" pre-requisite knowledge before proceding further. This is vital for any complicated knowledge disciplines.

\subparagraph{Glauret Integral}
The Glauret integral is a rather difficult integral. It also appears alot in many other areas of fluid mechanics, which is why different values of n in the integral expression has been carefully evaluated beforehand. This saves overhead when performing computations for the thin airfoil problem.

\end{document}
