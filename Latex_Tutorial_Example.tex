\documentclass[a4paper, 12pt]{report}

\usepackage{comment}
\usepackage{hyperref}
\usepackage{geometry}
\usepackage{listings}
\usepackage{xcolor}
\usepackage{amsmath}
%Needed For tables
\usepackage{array}

\geometry{portrait, margin= 0.43in}

\def\tabsize{1.9cm}
\def\ltabsize{1.4cm}

\definecolor{codegreen}{rgb}{0,0.6,0}
\definecolor{codegray}{rgb}{0.5,0.5,0.5}
\definecolor{codepurple}{rgb}{0.58,0,0.82}
\definecolor{backcolour}{rgb}{0.95,0.95,0.92}

\lstset{
	columns=fullflexible,
	frame=single,
	breaklines=true,
	backgroundcolor=\color{backcolour},   
	commentstyle=\color{codegreen},
	keywordstyle=\color{magenta},
	numberstyle=\tiny\color{codegray},
	stringstyle=\color{codepurple},
	basicstyle=\ttfamily\footnotesize,
	keepspaces=true,                 
	numbersep=5pt,                  
%	showspaces=true,               
%	showtabs=true,                  
	tabsize=2
}

\def\t{\theta}
\def\a{\alpha}
\def\be{\beta}
\def\w{\omega}
\def\la{\lambda}
\def\g{\gamma}
\def\f{\frac}
\def\l{\left}
\def\r{\right}
\def\dst{\displaystyle}
\def\b{\bar}
\def\h{\hat}
\def\ph{\phi}
\def\d{\cdot}
\def\n{\nabla}
\def\p{\partial}
\def\lap{\mathcal{L}}

\begin{document}

\title{LaTex Guide: Example-Based Approach}
\author{Ginger Gengar}
\date{$8^{th}$ May 2021}
\maketitle
\newpage

\tableofcontents
\newpage

\begin{center}
%Seperator
%Seperator
%Seperator
%Seperator
%Seperator
\section{Introduction}
\begin{comment}
\end{comment}

%Seperator
%Seperator
%Seperator
%Seperator
\subsection{Rationale} \label{rationale}
\begin{comment}
\end{comment}
Microsoft Word comes pre-installed in most machines and so most people use Microsoft word. The first thin about LaTex is that this typesetting system is not WYSIWYG (What You See Is What You Get). \texttt{LaTex} as a typesetting system separates the content from the actual form of the document, so that users can just focus on what the content of the document should contain first, and then worry about the document's appearance later using a bunch of \texttt{macros} or \texttt{environments}. I will explain in greater detail what those are. Just the separation between content and appearance is already a good thing as it encourages the user to tackle one problem at a time and not be overwhelmed. 
\\~\\The second aspect as to why \texttt{LaTex} is much better than word is that \texttt{LaTex} is \texttt{cli} based meanwhile word is \texttt{gui} based. \texttt{cli} here stands for "command line interface" meanwhile \texttt{gui} represents "graphical user interface". A \texttt{cli}  based typesetting system is much faster and convenient to write in compared to a \texttt{gui} system, though it might not seem like it at first. To write in \texttt{cli} one can just abandon the mouse and type all necessary commands. In \texttt{gui} however, one has to navigate to the specific option, scan the entire menu for the type of equation one wants to make and then select it and so forth. Not abandoning the mouse and having to click and search takes time and hinders speed and convenience of writing long mathematical equations. 
\\~\\Another advantage of \texttt{LaTex} is that it's beautiful, and neat. The equations in \texttt{LaTex} are written so beautifully compared to word. I will give you an example,
$$f(x) = \sum^{\infty}_{n = 0}\left[\frac{1}{n!}\frac{d^{n}f}{dx^{n}}(x-a)^{n}\right]$$
Try recreating something as neat as that in word. The equation above is quite standard and easy to produce in LaTex without even any further configurations. 

%Seperator
%Seperator
%Seperator
%Seperator
\subsection{Guide Usage}
\begin{comment}
\end{comment}
This guide is meant to be used for the absolute beginner to \texttt{LaTex}. The way this guide works is to show by example. The readers are HEAVILY ENCOURAGED to try the examples for themselves to learn how the \texttt{LaTex} typsetting system really works. This guide is meant to "teach by inferrence" wherein snippets of code is shown and then based on the results they produce, the readers can "infer" the rules of the \texttt{LaTex} typesetting system.

%Seperator
%Seperator
%Seperator
%Seperator
\subsection{Computing Environment}
\begin{comment}
\end{comment}
\texttt{LaTex} as a typesetting system works absolutely well with a standard \texttt{Linux} operating system, but also works fine for \texttt{Windows} system. Before I had migrated to \texttt{Linux} I used MikTex on windows. There are other \texttt{LaTex} distributions out there that can work. After installing \texttt{LaTex} using the package manager in \texttt{Linux} or through \texttt{MikTex} or some other distribution in windows, it is important to keep in mind that \texttt{LaTex} is compiled. Here are the steps to writing in \texttt{LaTex}:
\begin{enumerate}
\item Write code in \texttt{LaTex} using a text editor: On a linux system, one can choose their own text editors, such as \texttt{vim}, \texttt{emacs}, \texttt{nano}, \texttt{sublime}, just to name a few. These text editors work the same. All we are doing by invoking the text editor is writing some \texttt{ASCII} words into a file in a directory of our choosing.
\item Compile the \texttt{LaTex} code using a compiler (mine is \texttt{pdflatex}): The compilation process here typically uses the terminal. To compile a \texttt{.tex} file named \texttt{myfile.tex} using \texttt{pdflatex}, do this \texttt{pdflatex myfile.tex}.
\item Examine the generated \texttt{PDF} document, or, if an error occured, examine the erorr, fix the \texttt{LaTex} code and retry compiling at step $2$.
\end{enumerate}
When I first started I made alot of errors. If we make errors in not having balanced braces or invoked an invalid command, then the \texttt{LaTex} compiler will complain and no document is generated, because code is invalid. When I first started it was so difficult for me to interpret the common error messages. So it is advised to compile the document often, so that if an error were to occur, there is only a few changes before last compilation and it would help significantly in debugging \texttt{LaTex} code.
\\~\\Of course, this is the common beginner experience. As you use \texttt{LaTex} and gain more familiarity with it, you will naturally avoid all the common errors and writing \texttt{LaTex} rarely invokes any new errors, thus making \texttt{LaTex} a sane and effective means to communicate formal ideas.

%Seperator
%Seperator
%Seperator
%Seperator
%Seperator
\section{The Simplest Document}
\begin{comment}
\end{comment}
The simplest \texttt{LaTex} document that I can produce is shown below. The document below prints only "This is some text".
\lstinputlisting[language=tex]{Base.tex}
To me, it seems that only $3$ things are needed for the absolute minimum of a \texttt{LaTex} document:
\begin{enumerate}
\item Declaration at the top of what "Type" of Document one wants to generate: \texttt{documentclass[a4paper, 12pt]\{report\}}  
\item Environment block statements: \texttt{begin\{document\}}  followed by a \texttt{end\{document\}} 
\item Some small content inside the Environment block statement: \texttt{This is some text.} 
\end{enumerate}
Earlier I mentioned "macros" and "environments". An environment was invoked in the \texttt{LaTex} document above. The \texttt{LaTex} document above contains a single environment: \texttt{document}. 

%Seperator
%Seperator
%Seperator
%Seperator
%Seperator
\section{Commenting Lines}
\begin{comment}
\end{comment}
The base \texttt{LaTex} typesetting system acknowledges that a line beginning with \texttt{\%} contains comments. All lines beginning with a \texttt{\%} would be ignored by the \texttt{LaTex} compiler. An example is shown below,
\lstinputlisting[language=tex]{Comments_Base.tex}
You can make comments using the \texttt{\%} anywhere in a \texttt{LaTex} document. The example below is a valid \texttt{LaTex} document and demonstrates this fact:
\lstinputlisting[language=tex]{Comments_Anywhere.tex}
Comments are useful for encoding messages that are just for the author and not for the reader's to see. For example, when writing a long document, perhaps some part of the document needs to be made clearer or is wrong. It would be useful to add comments to mark where changes are necessary. The method shown above works great, except that it is limited to single line comments.

%Seperator
%Seperator
%Seperator
%Seperator
%Seperator
\section{Introduction to Packages}
\begin{comment}
\end{comment}
The most simple document we studied earlier is shown below,
\lstinputlisting[language=tex]{Base.tex}
Comments were added to mark the "Preamble" region and the "Document Body" region.
\lstinputlisting[language=tex]{Regions.tex}
The preamble is where packages are included for the rest of the document. Packages extends the commands a programmer can use, thereby extending \texttt{LaTex} functionalities. Earlier, I mentioned how it is only possible to comment single lines with the \texttt{\%} command. To make multi-line commented blocks, we need to use the \texttt{comment} package in order to declare multi-line commented blocks. An example of how to load the package is shown below,
\lstinputlisting[language=tex]{Multi_Block_Comments.tex}
Note in the example above, all we have done is load the \texttt{comment} package. We have not used the \texttt{comment} package yet. The \texttt{comment} package adds a bunch of utilities, but most importantly it declares a new environment: \texttt{comment}. Environments were introduced briefly and previously. Below shows a demonstration of how to use the \texttt{comment} environment in making multi-line block comments:
\lstinputlisting[language=tex]{Multi_Block_Comments_Usage.tex}
The statement \texttt{usepackage\{comment\}} is necessary. Without this, the \texttt{LaTex} document would not compile. Try commenting that statement out and try compiling and see what happens. The comment environment must be used after the \texttt{usepackage} statement. This is because using the \texttt{LaTex} compilers read the documents sequentially and would not recognize the environment before it is loaded.

%Seperator
%Seperator
%Seperator
%Seperator
%Seperator
\section{Document Margins}
\begin{comment}
\end{comment}
The document below is just a simple document that contains some text. The text shown below is long enough to fit easily into multiple lines. 
\lstinputlisting[language=tex]{Margins.tex}
Try compiling it. The document as is kind of looks "stifled" the margins are kind of too wide and it would be great if the margins could be made thinner. This is exactly what the \texttt{geometry} package allows us to do. Consider an addition of a few lines in the pre-amble as shown below,
\lstinputlisting[language=tex]{Margins_Thin.tex}
The statement \texttt{usepackage\{geometry\}} loads the geometry package and the statement \texttt{geometry\{portrait, margin= 0.8in\}} determines what the size of the margins are going to be and also the layout of the document. Try changing the margin value and the configuration to \texttt{landscape} from \texttt{portrait}. 

%Seperator
%Seperator
%Seperator
%Seperator
%Seperator
\section{Basic Equations}
\begin{comment}
\end{comment}
Consider the expression $x^{2} = 3$. Let us try to write that in \texttt{LaTex}. Begin by writing this inside the body of a \texttt{LaTex} document:
\begin{lstlisting}[language=tex]
\begin{equation}
x^{2} = 3
\end{equation}
\end{lstlisting}
The output of the code above is shown below. 
\begin{equation}
x^{2} = 3
\end{equation}
Here we are invoking a basic \texttt{LaTex} environment \texttt{equation}. There is a primitive command in \texttt{Tex} for making simple equations,
\begin{lstlisting}[language=Tex]
$$x^{2} = 3$$
\end{lstlisting}
The output of the primitive \texttt{LaTex} code is shown below,
$$x^{2} = 3$$
This has certain issues possibly with inconsisten vertical spacing, so it is recommended to do this instead,
\begin{lstlisting}[language=Tex]
\[x^{2} = 3\]
\end{lstlisting}
The output of the \texttt{LaTex} code is shown below,
\[x^{2} = 3\]
It really depends on personal preference, but I've used the \texttt{\$\$ \$\$} command and it has worked so far. Another important thing to note is that spaces do not matter in a \texttt{LaTex} \texttt{equation} environment. Here is an example,
\begin{lstlisting}[language=tex]
$$x = ab$$
\end{lstlisting}
The output of the \texttt{LaTex} code of the product of $2$ numbers without space is shown below,
$$x = ab$$
Now adding alot of space between $a$ and $b$,
\begin{lstlisting}[language=tex]
$$x = a            b$$
\end{lstlisting}
The result of the code above is shown below,
$$x = a            b$$
This shows that spacing in the variables do not matter in the \texttt{equation} environment or has no effect on how the equation is interpreted by the \texttt{LaTex} compiler.

%Seperator
%Seperator
%Seperator
%Seperator
%Seperator
\section{Amsmath Package}
\begin{comment}
\end{comment}
The \texttt{amsmath} package is massive package that extends the mathematical typesetting capabilities of \texttt{LaTex}. The \texttt{amsmath} package can be loaded using the command
\begin{lstlisting}[language=tex]
\usepackage{amsmath}
\end{lstlisting}
\texttt{amsmath} is such an important package that almost any \texttt{LaTex} document that has any substantial mathematical expressions would need to use \texttt{amsmath}.

%Seperator
%Seperator
%Seperator
%Seperator
\subsection{Suppressing Equation Labels}
\begin{comment}
\end{comment}

The \texttt{amsmath} package overwrite some of the basics of the \texttt{equation} environment. Adding a \texttt{*} symbol on the \texttt{equation} environment suppresses the labellings. The example below would demonstrate this:
\begin{lstlisting}[language=tex]
\begin{equation*}
x^{2} = 3
\end{equation*}
\end{lstlisting}
The output of the code above is shown below,
\begin{equation*}
x^{2} = 3
\end{equation*}
Why would we ever want to suppress the labelling? Imagine you were writing a long mathematical derivation, and you wrote every single equation with labelling unsuppressed. The last line would be labelled with some obscenely high number due to the sheer amount of steps that was done in the derivation. So it is unruly to see meaningless labels and is best to suppress the labels except for the time the equations need to be referred back to. 

%Seperator
%Seperator
%Seperator
%Seperator
\subsection{Aligning Equations}
\begin{comment}
\end{comment}

%Seperator
%Seperator
%Seperator
%Seperator
\subsection{Multi-Equation Lines} \label{multi-equation line}
\begin{comment}
\end{comment}
Imagine that you want to communicate a bunch of simple equations. Putting the each equation on a newline, is the default, but that would leave alot of wasted space. Here is an example of what I mean:
\begin{lstlisting}[language=tex]
$$x = k$$
$$y = q$$
$$z = u$$
$$b = m$$
\end{lstlisting}
The outuput of the \texttt{LaTex} code above is shown below,
$$x = k$$
$$y = q$$
$$z = u$$
$$b = m$$
It would be far more compact to write those simple equation in a single line. A naive approach of that would be to do this:
\begin{lstlisting}[language=tex]
$$x = k  y = q  z = u  b = m$$
\end{lstlisting}
The output of the naive approach is shown below,
$$x = k  y = q  z = u  b = m$$
The equations look completely messed up, because there are no gaps between them. The \texttt{amsmath} package provides a way to add gaps between equations, using the command \texttt{\textbackslash quad}. A more reasonable approach is to add \texttt{\textbackslash quad} in between the equations, like this:
\begin{lstlisting}[language=tex]
$$x = k \quad y = q \quad z = u \quad b = m$$
\end{lstlisting}
The output of the \texttt{LaTex} code with \texttt{\textbackslash quad} is shown below,
$$x = k \quad y = q \quad z = u \quad b = m$$
The result looks okay, but it could be made better by adding a comma in between to separate the equations,
\begin{lstlisting}[language=tex]
$$x = k \quad, y = q \quad, z = u \quad, b = m$$
\end{lstlisting}
The output of the \texttt{LaTex} equation above is shown below,
$$x = k \quad, y = q \quad, z = u \quad, b = m$$
My personal preference looks something like this instead,
\begin{lstlisting}[language=tex]
$$x = k \quad,\quad y = q \quad,\quad z = u \quad,\quad b = m$$
\end{lstlisting}
$$x = k \quad,\quad y = q \quad,\quad z = u \quad,\quad b = m$$

%Seperator
%Seperator
%Seperator
%Seperator
%Seperator
\section{Macros}
\begin{comment}
Macros are evil
\end{comment}
Macros are a set of commands that allow the programmer to redefine typical \texttt{LaTex} command as something else. These commands are typically declared in the pre-amble of a document. 

%Seperator
%Seperator
%Seperator
%Seperator
\subsection{Usage \& Syntax}
\begin{comment}
\end{comment}
Consider the following command declared in the pre-amble of a document:
\begin{lstlisting}[language=tex]
\def\t{\theta}
\end{lstlisting}
The command \texttt{\textbackslash def} is a special keyword, which allows the command inside the brackets, in this case \texttt{\textbackslash theta} to be redefined as the command in front of the bracket, which is \texttt{\textbackslash t}. Therefore, when the command shown below is invoked.
\begin{lstlisting}[language=tex]
$$\t$$
\end{lstlisting}
The output of the \texttt{LaTex} code shown above,
$$\t$$
Here is a list of Macros I had personally used in the past,
\begin{lstlisting}[language=tex]
\def\t{\theta}
\def\a{\alpha}
\def\be{\beta}
\def\w{\omega}
\def\la{\lambda}
\def\g{\gamma}
\def\f{\frac}
\def\l{\left}
\def\r{\right}
\def\dst{\displaystyle}
\def\b{\bar}
\def\h{\hat}
\def\ph{\phi}
\def\d{\cdot}
\def\n{\nabla}
\def\p{\partial}
\def\lap{\mathcal{L}}
\end{lstlisting}
With the macros defined as such, code as shown below,
\begin{lstlisting}[language=tex]
$$\t \quad,\quad \a \quad,\quad \ph \quad,\quad \lap$$
\end{lstlisting}
compiles into an equation shown below,
$$\t \quad,\quad \a \quad,\quad \ph \quad,\quad \lap$$

%Seperator
%Seperator
%Seperator
%Seperator
\subsection{Advantages and Disadvantages}
\begin{comment}
\end{comment}
Should anyone use Macros? Macros are one way to write in \texttt{LaTex} faster. By redefining an otherwise long command such as \texttt{\textbackslash frac\{\}\{\}} one can simply write \texttt{\textbackslash f\{\}\{\}}, which improves speed of writing a \texttt{LaTex} document. This is not too obvious since it only saves $3$ keystrokes, but what about \texttt{\textbackslash left} as \texttt{\textbackslash l}  and \texttt{\textbackslash right} as \texttt{\textbackslash r}? Those macros save $4$ keystrokes for \texttt{right} and $3$ keystrokes for \texttt{left}. The small increments in ease stacks up when writing increasingly long documents.
\\~\\That being said, I believe that macros are evil. The issue with macros is that the commands we redefine is author specific. If Sue redefines \texttt{\textbackslash displaystyle} as \texttt{\textbackslash dst}, everytime Mary looks at \texttt{\textbackslash displaystyle} she will get confused because \texttt{\textbackslash displaystyle} is not part of the standard \texttt{LaTex} command. If Sue and Mary work together, then Mary would have to keep referring to the preamble to translate the Macros, or Mary and Sue must work in completely different parts of the document and cannot easily check each other's work. This is just for a team of $2$. But what happens if a project is sufficiently large and has $7$ people working on it? Does it mean we need $7$ different Macros redefining \texttt{\textbackslash frac\{\}\{\}}? It is silly and that is why macros should be avoided.
\\~\\So is there another way? Yes. There are cleaner ways to reduce keystrokes while typing. The way to reduce keystrokes is by using \texttt{snippets}. \texttt{snippets} allow a text editor such as \texttt{vim} to expand a string into another string, and \texttt{snippets} are configured by the author. For example, in my \texttt{snippet} setup, \texttt{dst} will automatically expand to \texttt{\textbackslash displaystyle}. Therefore, I can enjoy the same benefits of reduced keystrokes while writing, while retaining readability by using just the standard \texttt{LaTex} commands.
\\~\\There are other ways as well that make Macros obselete. For example, \texttt{vim} has a plugin named \texttt{vimtex} which allows the adding of the \texttt{\textbackslash left} and \texttt{\textbackslash right} commands to a set of brackets by just keying \texttt{tsd}, which allows for even more effective writing.

%Seperator
%Seperator
%Seperator
%Seperator
%Seperator
\section{Common Mathematical Equations}
\begin{comment}
\end{comment}
Now the "fun parts" of mathematical writing in \texttt{LaTex} begins.

%Seperator
%Seperator
%Seperator
%Seperator
\subsection{Greek Symbols}
\begin{comment}
\end{comment}
A list of the greek symbols that are available using the base \texttt{LaTex} is shown below. Some of the lowercase greek symbols shown below do not have an uppercase equivalent. This is because the uppercase equivalent is just a typical latin alphabet capital. For example. \textbackslash \texttt{alpha} is valid, but \textbackslash \texttt{Alpha} is not valid. It seems that if one wants to represent \textbackslash \texttt{Alpha} one can just invoke \$A\$, which results in $A$.
\\~\\\begin{tabular}{|m{\tabsize}|m{\ltabsize}|m{\tabsize}|m{\ltabsize}|m{\tabsize}|m{\ltabsize}|m{\tabsize}|m{\ltabsize}|}
\hline
Command & Symbol & Command & Symbol & Command & Symbol & Command & Symbol \\ \hline
\textbackslash \texttt{alpha}  & $\alpha$ & \textbackslash \texttt{beta}  & $\beta$ & \textbackslash \texttt{gamma}  & $\gamma$ & \textbackslash \texttt{Gamma} & $\Gamma$ \\ \hline
\textbackslash \texttt{delta}  & $\delta$ & \textbackslash \texttt{Delta} & $\Delta$ & \textbackslash \texttt{epsilon} & $\epsilon$ & \textbackslash \texttt{zeta} & $\zeta$ \\ \hline
\textbackslash \texttt{eta} & $\eta$ & \textbackslash \texttt{theta} & $\theta$ & \textbackslash \texttt{Theta}  & $\Theta$ & \textbackslash \texttt{iota} & $\iota$ \\ \hline
\textbackslash \texttt{kappa}  & $\kappa$ & \textbackslash \texttt{lambda} & $\lambda$ & \textbackslash \texttt{Lambda} & $\Lambda$ & \textbackslash \texttt{mu}  & $\mu$ \\ \hline
\textbackslash \texttt{nu} & $\nu$ & \textbackslash \texttt{xi} & $\xi$ & \textbackslash \texttt{Xi} & $\Xi$ & \textbackslash \texttt{pi} & $\pi$ \\ \hline
\textbackslash \texttt{Pi} & $\Pi$ & \textbackslash \texttt{rho} & $\rho$ & \textbackslash \texttt{sigma} & $\sigma$ & \textbackslash \texttt{Sigma} & $\Sigma$ \\ \hline
\textbackslash \texttt{tau}  & $\tau$ & \textbackslash \texttt{upsilon} & $\upsilon$ & \textbackslash \texttt{Upsilon} & $\Upsilon$ & \textbackslash \texttt{phi} & $\phi$ \\ \hline
\textbackslash \texttt{Phi} & $\Phi$ & \textbackslash \texttt{chi} & $\chi$ & \textbackslash \texttt{psi} & $\psi$ & \textbackslash \texttt{Psi} & $\Psi$ \\ \hline
\textbackslash \texttt{omega} & $\omega$ & \textbackslash \texttt{Omega} & $\Omega$ & None & None & None & None \\ \hline
\end{tabular}

%Seperator
%Seperator
%Seperator
%Seperator
\subsection{Subscripts \& Superscripts} \label{subscript superscript}
\begin{comment}
\end{comment}
Often times, it is important to make a distinction between variables. The subscript is used to make such a distinction. This is an example of how to use a subscript. \texttt{LaTex} should print $x$ subscripted by $a$.
\begin{lstlisting}[language=tex]
$$x_{a}$$
\end{lstlisting}
The output of the \texttt{LaTex} code above,
$$x_{a}$$
Earlier, I have already shown how to do a superscript, though I did not formally declare how to use superscript. The example below is going to print $x$ superscripted by $a$.
\begin{lstlisting}[language=tex]
$$x^{a}$$
\end{lstlisting}
The output of the superscript is shown below,
$$x^{a}$$
The subscripts and superscripts accept whatever is inside the brackets. Below is an example that demonstrates putting whatever we want inside the brackets,
\begin{lstlisting}[language=tex]
$$x^{abc/d} \quad,\quad x_{abc/d}$$
\end{lstlisting}
Please do not be intimidated by the \texttt{\textbackslash quad,\textbackslash quad} part. This was explained in section $\ref{multi-equation line}$. This command only adds a space followed by a \texttt{,} and then another space. The output of the \texttt{LaTex} code above is shown below,
$$x^{abc/d} \quad,\quad x_{abc/d}$$
The subscripts and superscripts can be used at the same time in the same equation. The \texttt{LaTex} code below will show some a polynomial expression demonstrating how subscripts and superscripts can be used at the same time,
\begin{lstlisting}[language=tex]
$$0 = k_{a}^{2}x_{i}^{3} + m_{s}^{3}x_{i}^{2} + l_{r}^{2}x_{i} + k_{b}$$
\end{lstlisting}
The output of the \texttt{LaTex} code is shown below,
$$0 = k_{a}^{2}x_{i}^{3} + m_{s}^{3}x_{i}^{2} + l_{r}^{2}x_{i} + k_{b}$$
Note that when we the order at which a subscript and superscript is written for a variable does not matter. For example,
\begin{lstlisting}[language=tex]
$$k_{a}^{2} \quad,\quad k^{2}_{a}$$
\end{lstlisting}
The output of the \texttt{LaTex} code is shown below,
$$k_{a}^{2} \quad,\quad k^{2}_{a}$$
Subscripts and superscripts might be enough for most applications, but what if we wanted even more control of how a variable is specified? What if we wanted to add small subscripts and superscripts at the front of a variable? This might sound like a silly idea after all, it is a rarity to find any mathematical working with such an odd form of notation. I have encountered this problem before in particular in the fields of dynamics, wherein it was necessary to specify alot of information about a variable such as its basis vectors, the reference frame it is nested in, the object the quantity belongs to, and a counter variable. Consider the following \texttt{LaTex} code:
\begin{lstlisting}[language=tex]
$${}^{a}_{b}$$
\end{lstlisting}
The output of the strange \texttt{LaTex} code is shown below,
$${}^{a}_{b}$$
So the \texttt{LaTex} code above subscripts an empty set by $b$ and superscripts the same empty set by $a$. Since the empty set represented by \texttt{\{\}} does not represent anything, the compiler prints $a$ on top of $b$. So the code above can be used to give forward subscript and superscript to some variable $v$. The code below demonstrates this,
\begin{lstlisting}[language=tex]
$${}^{a}_{b}v$$
\end{lstlisting}
The \texttt{LaTex} code prints,
$${}^{a}_{b}v$$
From then on, it is simple enough to add the conventional superscripts and subscripts to the variable to give the variable $v$ a combined total of $4$ specifiers, $a$, $b$, $c$ and $d$ :
\begin{lstlisting}[language=tex]
$${}^{a}_{b}v^{c}_{d}$$
\end{lstlisting}
The \texttt{LaTex} code above produces the result below,
$${}^{a}_{b}v^{c}_{d}$$



%Seperator
%Seperator
%Seperator
%Seperator
\subsection{Fractions \& Brackets} \label{fractions part}
\begin{comment}
\end{comment}
Usually there are $3$ levels of conventional mathematical brackets, those are demonstrated below,
$$\{[()]\}$$
The most common bracket is generated by the code shown below,
\begin{lstlisting}[language=tex]
$$()$$
\end{lstlisting}
The output of the \texttt{LaTex} code is shown below,
$$()$$
The $2^{nd}$ level of bracket is generated by the code below,
\begin{lstlisting}[language=tex]
$$[]$$
\end{lstlisting}
The output of the \texttt{LaTex} code is shown below,
$$[]$$
The outermost bracket is generated by the code below,
\begin{lstlisting}[language=tex]
$$\{\}$$
\end{lstlisting}
The output of the \texttt{LaTex} code is shown below,
$$\{\}$$
The last bracket is kind of unique. The issue with just commanding \texttt{\{\}} is that the characters \texttt{\{} and \texttt{\}} are special characters in the \texttt{LaTex} typesetting system. Therefore, \textbackslash must be used in front to specify the bracket characters as truly brackets and not special \texttt{LaTex} characters.
The way to write fractions in \texttt{LaTex} is to invoke the \texttt{\textbackslash frac\{\}\{\}} command. The $1^{st}$ set of brackets would be the numerator meanwhile the $2^{nd}$ set of brackets would be the denominator. An example of that is shown below,
\begin{lstlisting}[language=tex]
$$\frac{numerator}{denominator}$$
\end{lstlisting}
The resulting \texttt{LaTex} output is shown below,
$$\frac{numerator}{denominator}$$
Now it is quite typical to encounter a fraction nested within another fraction. In default \texttt{LaTex}, a situation like this would make the nested fraction usually smaller. An example of that is shown below,
\begin{lstlisting}[language=tex]
$$f = \frac{1}{1+\frac{3}{A}}$$
\end{lstlisting}
The output of the \texttt{LaTex} code is shown below,
$$f = \frac{1}{1+\frac{3}{A}}$$
Sometimes this is acceptable or desirable, but personally I find this ugly. So, to inflate the nested fraction to its original size, we can invoke the \texttt{\textbackslash displaystyle} command. Here is an example of the \texttt{\textbackslash displaystyle} command,
\begin{lstlisting}[language=tex]
$$f = \frac{1}{\displaystyle 1+\frac{3}{A}}$$
\end{lstlisting}
The corrected \texttt{LaTex} code output is shown below,
$$f = \frac{1}{\displaystyle 1+\frac{3}{A}}$$
Here, notice how the \texttt{\textbackslash displaystyle} command has returned the nested fraction to its original size. Do note that the \texttt{\textbackslash displaystyle} command needs to be declared in the region of the nested fraction, which in this case is the numerator of the "main" fraction. Often times, it is necessary to add brackets to fractions. Consider the normal fraction,
\begin{lstlisting}[language=tex]
$$x = \frac{1}{2}$$
\end{lstlisting}
The output of the \texttt{LaTex} code above is shown below,
$$x = \frac{1}{2}$$
What if we naively declared brackets on the fraction like below:
\begin{lstlisting}[language=tex]
$$x = (\frac{1}{2})$$
\end{lstlisting}
The output of the \texttt{LaTex} code above is shown below,
$$x = (\frac{1}{2})$$
The brackets do not span the entire height of the fraction and this looks very ugly. So we can fix this by invoking the \texttt{\textbackslash left} and \texttt{\textbackslash right} on the brackets like this,
\begin{lstlisting}[language=tex]
$$x = \left(\frac{1}{2}\right)$$
\end{lstlisting}
The output of the \texttt{LaTex} code above is shown below,
$$x = \left(\frac{1}{2}\right)$$
The commands \texttt{\textbackslash left} and \texttt{\textbackslash right} is also compatible for other types of brackets as well, here is an example:
\begin{lstlisting}[language=tex]
$$x = \left[\frac{1}{2}\right]$$
\end{lstlisting}
The output of the \texttt{LaTex} code above is shown below,
$$x = \left[\frac{1}{2}\right]$$
\begin{lstlisting}[language=tex]
$$x = \left\{\frac{1}{2}\right\}$$
\end{lstlisting}
The output of the \texttt{LaTex} code above is shown below,
$$x = \left\{\frac{1}{2}\right\}$$
the \texttt{\textbackslash left} and \texttt{\textbackslash right} command combined together with the previous \texttt{\textbackslash displaystyle} example:
\begin{lstlisting}[language=tex]
$$f = \left\{\left[\left(\frac{1}{\displaystyle 1+\frac{3}{A}}\right)\right]\right\}$$
\end{lstlisting}
The output of the \texttt{LaTex} code above is shown below,
$$f = \left\{\left[\left(\frac{1}{\displaystyle 1+\frac{3}{A}}\right)\right]\right\}$$

%Seperator
%Seperator
%Seperator
%Seperator
\subsection{Basic Calculus Operations}
\begin{comment}
\end{comment}
%Seperator
%Seperator
%Seperator
\subsubsection{Derivatives} \label{derivatives}
\begin{comment}
\end{comment}

The code below demonstrates how to show a derivative operator,
\begin{lstlisting}[language=tex]
$$\frac{d}{dx}$$
\end{lstlisting}
The \texttt{LaTex} code above, produces the output shown below,
$$\frac{d}{dx}$$
Note that a simple derivative operator is just a fraction whose numerator is \texttt{d} and denominator is \texttt{d var} wherein you can chooose \texttt{var} the name of a variable of your choosing. To demonstrate the different choices of variables,
\begin{lstlisting}[language=tex]
$$\frac{d}{dx} \quad,\quad \frac{d}{dy} \quad,\quad \frac{d}{dz} \quad,\quad \frac{d}{d \alpha} \quad,\quad \frac{d}{d \beta} \quad,\quad \frac{d}{d \gamma} \quad,\quad \frac{d}{d \xi}$$
\end{lstlisting}
The \texttt{LaTex} code above, produces the output shown below,
$$\frac{d}{dx} \quad,\quad \frac{d}{dy} \quad,\quad \frac{d}{dz} \quad,\quad \frac{d}{d \alpha} \quad,\quad \frac{d}{d \beta} \quad,\quad \frac{d}{d \gamma} \quad,\quad \frac{d}{d \xi}$$
Sometimes we want to take derivative of a variable or function by "inlining" it to the derivative operator, here is an example of how to do that,
\begin{lstlisting}[language=tex]
$$\frac{d f(x)}{dx}$$
\end{lstlisting}
The \texttt{LaTex} code above, produces the output shown below,
$$\frac{d f(x)}{dx}$$
By the syntax of \texttt{\textbackslash frac\{\}\{\}}, it is sufficient to just add the object that is differentiated to the numerator of the derivative fraction. Preferrably, we might want to separate the derivative operator and the object of derivative. This shows how,
\begin{lstlisting}[language=tex]
$$\frac{d}{dx}[f(x)]$$
\end{lstlisting}
The \texttt{LaTex} code above, produces the output shown below,
$$\frac{d}{dx}[f(x)]$$
Taking multiple derivatives consecutively can be done by a simple subscript to the derivative numerators and denominators like so,
\begin{lstlisting}[language=tex]
$$\frac{d^{2}y}{dx^{2}} \quad,\quad \frac{d^{3}y}{dx^{3}} \quad,\quad \frac{d^{4}y}{dx^{4}} \quad,\quad \frac{d^{n}y}{dx^{n}}$$
\end{lstlisting}
The \texttt{LaTex} code above, produces the output shown below,
$$\frac{d^{2}y}{dx^{2}} \quad,\quad \frac{d^{3}y}{dx^{3}} \quad,\quad \frac{d^{4}y}{dx^{4}} \quad,\quad \frac{d^{n}y}{dx^{n}}$$
Preferrably if one seeks to separate operators,
\begin{lstlisting}[language=tex]
$$\frac{d^{2}}{dx^{2}}[f(x)] \quad,\quad \frac{d^{3}}{dx^{3}}[f(x)] \quad,\quad \frac{d^{4}}{dx^{4}}[f(x)] \quad,\quad \frac{d^{n}}{dx^{n}}[f(x)]$$
\end{lstlisting}
The \texttt{LaTex} code above, produces the output shown below,
$$\frac{d^{2}}{dx^{2}}[f(x)] \quad,\quad \frac{d^{3}}{dx^{3}}[f(x)] \quad,\quad \frac{d^{4}}{dx^{4}}[f(x)] \quad,\quad \frac{d^{n}}{dx^{n}}[f(x)]$$
Suppose we had a fraction in object we would like to differentiate, the results of section $\ref{fractions part}$ would allow the brackets to span the entire fraction. Below is a demonstration
\begin{lstlisting}[language=tex]
$$\frac{d^{n}}{dx^{n}}[\frac{f(x)}{g(x)}] \quad,\quad \frac{d^{n}}{dx^{n}}\left[\frac{f(x)}{g(x)}\right]$$
\end{lstlisting}
The \texttt{LaTex} code above, produces the output shown below,
$$\frac{d^{n}}{dx^{n}}[\frac{f(x)}{g(x)}] \quad,\quad \frac{d^{n}}{dx^{n}}\left[\frac{f(x)}{g(x)}\right]$$
The one on the left is without the \texttt{\textbackslash left} and \texttt{\textbackslash right} commands. The brackets are stunted. THe right invokes the \texttt{\textbackslash left} and \texttt{\textbackslash right} command, the brackets are appropriately sized. So the findings from the previous sections can be reused in conjunction with each other to form increasingly complicated expressions with simple commands. It is also possible to change the total derivative with the partial derivative. Instead of using \texttt{d} use the command \texttt{\textbackslash partial}. Below demonstrates how to do partial derivatives instead of normal derivatives,
\begin{lstlisting}[language=tex]
$$\frac{\partial^{2}}{\partial x^{2}}[f(x)] \quad,\quad \frac{\partial^{3}}{\partial x^{3}}[f(x)] \quad,\quad \frac{\partial^{4}}{\partial x^{4}}[f(x)] \quad,\quad \frac{\partial^{n}}{\partial x^{n}}[f(x)]$$
\end{lstlisting}
The \texttt{LaTex} code above, produces the output shown below,
$$\frac{\partial^{2}}{\partial x^{2}}[f(x)] \quad,\quad \frac{\partial^{3}}{\partial x^{3}}[f(x)] \quad,\quad \frac{\partial^{4}}{\partial x^{4}}[f(x)] \quad,\quad \frac{\partial^{n}}{\partial x^{n}}[f(x)]$$

%Seperator
%Seperator
%Seperator
\subsubsection{Integrals}
\begin{comment}
\end{comment}
The integral symbol could be invoked using the \texttt{\textbackslash int} command. Below is a demonstration
\begin{lstlisting}[language=tex]
$$\int$$
\end{lstlisting}
The \texttt{LaTex} code above, produces the output shown below,
$$\int$$
Suppose we wanted to take a simple indefinite integral of a monomial $x^{2}$, then we can use the \texttt{\textbackslash int} command shown previously, followed by the monomial, and then ended by \texttt{d var} wherein \texttt{var} is the variable we are integrating with respect to. Below shows a demonstration of this,
\begin{lstlisting}[language=tex]
$$\int x^{2}dx$$
\end{lstlisting}
The \texttt{LaTex} code above, produces the output shown below,
$$\int x^{2}dx$$
The integral looks decent, except that the integrand which is $x^{2}$ in this case, has no gap with the end of the integral, $dx$. To remedy this, we can invoke the base \texttt{LaTex} command \texttt{\textbackslash ,} to add a gap between the integrand and the end of the integral like so,
\begin{lstlisting}[language=tex]
$$\int x^{2}\,dx$$
\end{lstlisting}
The \texttt{LaTex} code above, produces the output shown below,
$$\int x^{2}\,dx$$
This looks much better. After the indefinite integral, how to write a definite integral with bounds? To do that, we can just superscript and subscript the \texttt{\textbackslash int} command to fill the bounds of the integral. Below is a demonstration of this superscripting and subscripting,
\begin{lstlisting}[language=tex]
$$\int^{a}_{b} x^{2}\,dx$$
\end{lstlisting}
The \texttt{LaTex} code above, produces the output shown below,
$$\int^{a}_{b} x^{2}\,dx$$
Of course as with all things in \texttt{LaTex} the integral commands can be combined together to form more complex expressions such as the one below,
\begin{lstlisting}[language=tex]
$$\int^{e}_{f} \int^{c}_{d} \int^{a}_{b} g(x,y,z) \,dxdydz$$
\end{lstlisting}
The \texttt{LaTex} code above, produces the output shown below,
$$\int^{e}_{f} \int^{c}_{d} \int^{a}_{b} g(x,y,z) \,dxdydz$$
Here is another example, this time for cylindrical coordinates, just to demonstrate the \texttt{LaTex} syntax,
\begin{lstlisting}[language=tex]
$$\int^{h}_{0}\int^{2\pi}_{0}\int^{r_{2}}_{r_{1}} f(r,\theta,z) \,rdr\,d\theta\,dz$$
\end{lstlisting}
The \texttt{LaTex} code above, produces the output shown below,
$$\int^{h}_{0}\int^{2\pi}_{0}\int^{r_{2}}_{r_{1}} f(r,\theta,z) \,rdr\,d\theta\,dz$$

%Seperator
%Seperator
%Seperator
\subsubsection{Summation Operations}
\begin{comment}
\end{comment}
Earlier I showed how to write the greek symbol \texttt{\textbackslash Sigma}, which compiles to $\Sigma$. This is tiny so don't do this. The true summation operation is invoked using the \texttt{\textbackslash sum} command. Here is a demonstration,
\begin{lstlisting}[language=tex]
$$\sum$$
\end{lstlisting}
The \texttt{LaTex} code above, produces the output shown below,
$$\sum$$
The summation operation is just a symbol, similar to the way the integral symbol works. To do say summation of some coefficient $a_{i}$:
\begin{lstlisting}[language=tex]
$$\sum(a_{i})$$
\end{lstlisting}
The \texttt{LaTex} code above, produces the output shown below,
$$\sum(a_{i})$$
Other types of brackets can be used as well,
\begin{lstlisting}[language=tex]
$$\sum[a_{i}]$$
\end{lstlisting}
The \texttt{LaTex} code above, produces the output shown below,
$$\sum[a_{i}]$$
If the object to be summed over is a fraction, then the commands \texttt{\textbackslash left} and \texttt{\textbackslash right} would both be valid as well. Here is an example,
\begin{lstlisting}[language=tex]
$$\sum\left[\frac{a_{i}}{i!}\right]$$
\end{lstlisting}
The \texttt{LaTex} code above, produces the output shown below,
$$\sum\left[\frac{a_{i}}{i!}\right]$$
This is fine for more basic equations, but we can add the lower bounds and upper bounds of the summation just like in the case with the integral. Here is an example,
\begin{lstlisting}[language=tex]
$$\sum^{n}_{i = 0}\left[\frac{a_{i}}{i!}\right]$$
\end{lstlisting}
The \texttt{LaTex} code above, produces the output shown below,
$$\sum^{n}_{i = 0}\left[\frac{a_{i}}{i!}\right]$$
Let us try reconstructing the taylor series shown earlier in section $\ref{rationale}$,
$$f(x) = \sum^{\infty}_{n = 0}\left[\frac{1}{n!}\frac{d^{n}f}{dx^{n}}(x-a)^{n}\right]$$
From section $\ref{subscript superscript}$, the polynomial part of the taylor series can be written,
\begin{lstlisting}[language=tex]
$$(x-a)^{n}$$
\end{lstlisting}
The \texttt{LaTex} code above, produces the output shown below,
$$(x-a)^{n}$$
From section $\ref{derivatives}$, we can easily construct the first order derivative of function $f$ using fractions and \texttt{d},
\begin{lstlisting}[language=tex]
$$\frac{df}{dx}$$
\end{lstlisting}
The \texttt{LaTex} code above, produces the output shown below,
$$\frac{df}{dx}$$
To take $n$ successive derivatives, just use the superscript notations, 
\begin{lstlisting}[language=tex]
$$\frac{d^{n}f}{dx^{n}}$$
\end{lstlisting}
The \texttt{LaTex} code above, produces the output shown below,
$$\frac{d^{n}f}{dx^{n}}$$
Putting these $2$ together forms,
\begin{lstlisting}[language=tex]
$$\frac{d^{n}f}{dx^{n}}(x-a)^{n}$$
\end{lstlisting}
The \texttt{LaTex} code above, produces the output shown below,
$$\frac{d^{n}f}{dx^{n}}(x-a)^{n}$$
Just adding a simple fraction at the beginning with a numerator \texttt{1} and denominator \texttt{n!} yields,
\begin{lstlisting}[language=tex]
$$\frac{1}{n!}\frac{d^{n}f}{dx^{n}}(x-a)^{n}$$
\end{lstlisting}
The \texttt{LaTex} code above, produces the output shown below,
$$\frac{1}{n!}\frac{d^{n}f}{dx^{n}}(x-a)^{n}$$
There's only $1$ additional new symbol that one has to take care of, the infinity symbol. Here is the syntax for the infinity symbol,
\begin{lstlisting}[language=tex]
$$\infty$$
\end{lstlisting}
The \texttt{LaTex} code above, produces the output shown below,
$$\infty$$
Putting all that inside a summation operation,
$$f(x) = \sum^{\infty}_{n = 0}\left[\frac{1}{n!}\frac{d^{n}f}{dx^{n}}(x-a)^{n}\right]$$
This is the taylor expression shown in section $\ref{rationale}$. 

%Seperator
%Seperator
%Seperator
\subsubsection{Product Operations}
\begin{comment}
\end{comment}
The product operators are similar to the integral operators and the summation operator. Here is an example,
\begin{lstlisting}[language=tex]
$$\prod^{n}_{i = 0}[a_{i}]$$
\end{lstlisting}
The \texttt{LaTex} code above, produces the output shown below,
$$\prod^{n}_{i = 0}[a_{i}]$$
The same rules for summation operations also apply for product operations. To find the successive multiplications of fraction objects, just do the same thing with the summation case,
\begin{lstlisting}[language=tex]
$$\prod^{n}_{i = 0}\left[\frac{a_{i}}{i!}\right]$$
\end{lstlisting}
The \texttt{LaTex} code above, produces the output shown below,
$$\prod^{n}_{i = 0}\left[\frac{a_{i}}{i!}\right]$$


%Seperator
%Seperator
%Seperator
%Seperator
\subsection{Vector Calculus Notations}
\begin{comment}
\end{comment}
The bar vector is typically used to denote that a vector. The way the bar is used is shown below,
\begin{lstlisting}[language=tex]
$$\bar{v}$$
\end{lstlisting}
The \texttt{LaTex} code above, produces the output shown below,
$$\bar{v}$$
A bar would be drawn over the entirety of whatever is in between \texttt{\{} and \texttt{\}}. Here is an example of using multiple symbols,
\begin{lstlisting}[language=tex]
$$\bar{vb}$$
\end{lstlisting}
The \texttt{LaTex} code above, produces the output shown below,
$$\bar{vb}$$
The hat is usually used to denote unit vectors. Its usage and syntax is similar to the bar case,
\begin{lstlisting}[language=tex]
$$\hat{v}$$
\end{lstlisting}
The \texttt{LaTex} code above, produces the output shown below,
$$\hat{v}$$
For multiple symbols,
\begin{lstlisting}[language=tex]
$$\hat{vb}$$
\end{lstlisting}
The \texttt{LaTex} code above, produces the output shown below,
$$\hat{vb}$$
Of course, in \texttt{LaTex}, anything can be mixed with each other. So, one can use a subscript and a bar or hat to represent some vector like this,
$$\bar{v_{i}} \quad,\quad \hat{v_{i}}$$
When discussing vector calculus, one cannot leave out the operations. Most of the operations are essentially symbols. To do a cross product notation, use \texttt{\textbackslash times} as shown below,
\begin{lstlisting}[language=tex]
$$\bar{a} \times \bar{b}$$
\end{lstlisting}
The \texttt{LaTex} code above, produces the output shown below,
$$\bar{a} \times \bar{b}$$
To do a dot product notation, use \texttt{\textbackslash cdot} as shown below,
\begin{lstlisting}[language=tex]
$$\bar{a} \cdot \bar{b}$$
\end{lstlisting}
The \texttt{LaTex} code above, produces the output shown below,
$$\bar{a} \cdot \bar{b}$$
To use the gradient operator, invoke the command \texttt{\textbackslash nabla} as shown below
\begin{lstlisting}[language=tex]
$$\nabla \bar{v}$$
\end{lstlisting}
The \texttt{LaTex} code above, produces the output shown below,
$$\nabla \bar{v}$$
Since commands can be mixed in \texttt{LaTex} it is also possible to get the curl and divergence of a vector field by using a combination of the operators above,
\begin{lstlisting}[language=tex]
$$\nabla \times \bar{v} \quad,\quad \nabla \cdot \bar{v}$$
\end{lstlisting}
The \texttt{LaTex} code above, produces the output shown below,
$$\nabla \times \bar{v} \quad,\quad \nabla \cdot \bar{v}$$

%Seperator
%Seperator
%Seperator
%Seperator
\subsection{Matrices}
\begin{comment}
\end{comment}
Matrices in \texttt{LaTex} are characterized by the type of brackets that they have. The \texttt{pmatrix} environment declared matrices with rounded brackets meanwhile the \texttt{bmatrix} environment declares matrices with sharp brackets. There are other different types of matrices other than \texttt{pmatrix} and \texttt{bmatrix}, but for now these $2$ should suffice. To write the next horizontal element of a matrix one use \texttt{\&}. Here is a simple example,
\begin{lstlisting}[language=tex]
$$\begin{bmatrix}
v_{1} & v_{2} & v_{3}
\end{bmatrix}$$
\end{lstlisting}
The \texttt{LaTex} code above, produces the output shown below,
$$\begin{bmatrix}
v_{1} & v_{2} & v_{3}
\end{bmatrix}$$
To have a larger gap between the columns of the matrix, one can use \texttt{\&\&} instead of just \texttt{\&}. Here is an example,
\begin{lstlisting}[language=tex]
$$\begin{bmatrix}
v_{1} && v_{2} && v_{3}
\end{bmatrix}$$
\end{lstlisting}
The \texttt{LaTex} code above, produces the output shown below,
$$\begin{bmatrix}
v_{1} && v_{2} && v_{3}
\end{bmatrix}$$
To write the next vertical element of a matrix, one can use \texttt{\textbackslash \textbackslash}. Here is a simple example,
\begin{lstlisting}[language=tex]
$$\begin{bmatrix}
v_{1} \\ v_{2} \\ v_{3}
\end{bmatrix}$$
\end{lstlisting}
The \texttt{LaTex} code above, produces the output shown below,
$$\begin{bmatrix}
v_{1} \\ v_{2} \\ v_{3}
\end{bmatrix}$$
To make the rows have a larger separation between them, one can use \texttt{\textbackslash \textbackslash \texttildelow \textbackslash \textbackslash} instead of just \texttt{\textbackslash \textbackslash}. Here is an example,
$$\begin{bmatrix}
v_{1} \\~\\ v_{2} \\~\\ v_{3}
\end{bmatrix}$$
Therefore, putting all that together allows us to write a simple matrix as shown below,
\begin{lstlisting}[language=tex]
$$\begin{bmatrix}
a && b \\
c && d
\end{bmatrix}$$
\end{lstlisting}
The \texttt{LaTex} code above, produces the output shown below,
$$\begin{bmatrix}
a && b \\
c && d
\end{bmatrix}$$
In principle, this matrix can be of any size. Therefore, a $3\times 3$ matrix could be written below,
\begin{lstlisting}[language=tex]
$$\begin{bmatrix}
v_{11} && v_{12} && v_{13} \\ 
v_{21} && v_{22} && v_{23} \\ 
v_{31} && v_{32} && v_{33} \\ 
\end{bmatrix}$$
\end{lstlisting}
The \texttt{LaTex} code above, produces the output shown below,
$$\begin{bmatrix}
v_{11} && v_{12} && v_{13} \\ 
v_{21} && v_{22} && v_{23} \\ 
v_{31} && v_{32} && v_{33} \\ 
\end{bmatrix}$$
We could also leave some entries blank and nothing will be printed for those elements. Here is an example to demonstrate,
\begin{lstlisting}[language=tex]
$$\begin{bmatrix}
v_{11} && v_{12} && v_{13} \\ 
v_{21} &&  && v_{23} \\ 
v_{31} && v_{32} && v_{33} \\ 
\end{bmatrix}$$
\end{lstlisting}
The \texttt{LaTex} code above, produces the output shown below,
$$\begin{bmatrix}
v_{11} && v_{12} && v_{13} \\ 
v_{21} &&  && v_{23} \\ 
v_{31} && v_{32} && v_{33} \\ 
\end{bmatrix}$$
\texttt{LaTex} matrices can also be superscripted and subscripted. Here is an example of that,
$$\begin{bmatrix}
a && b \\
c && d
\end{bmatrix}^{3}_{x}$$
A common mathematical statement regarding transposes therefore can already be written with the tools covered,
\begin{lstlisting}[language=tex]
$$\begin{bmatrix}
v_{1} && v_{2} && v_{3}
\end{bmatrix}^{T} = \begin{bmatrix}
v_{1} \\ v_{2} \\ v_{3}
\end{bmatrix}$$
\end{lstlisting}
The \texttt{LaTex} code above, produces the output shown below,
$$\begin{bmatrix}
v_{1} && v_{2} && v_{3}
\end{bmatrix}^{T} = \begin{bmatrix}
v_{1} \\ v_{2} \\ v_{3}
\end{bmatrix}$$
It is sometimes needed to speak about general-sized matrices, at which dots would be useful. To use horizontal dots in a matrix, one can just use the \texttt{\textbackslash dots} symbol as one of the matrix elements. Here is an example of this,
\begin{lstlisting}[language=tex]
$$\begin{bmatrix}
v_{1} & \dots & v_{n}
\end{bmatrix}$$
\end{lstlisting}
The \texttt{LaTex} code above, produces the output shown below,
$$\begin{bmatrix}
v_{1} & \dots & v_{n}
\end{bmatrix}$$
to invoke vertical dots this time, one can use the \texttt{\textbackslash vdots} symbol as one of the matrix elements, such as shown below,
\begin{lstlisting}[language=tex]
$$\begin{bmatrix}
v_{1} \\ \vdots \\ v_{n}
\end{bmatrix}$$
\end{lstlisting}
The \texttt{LaTex} code above, produces the output shown below,
$$\begin{bmatrix}
v_{1} \\ \vdots \\ v_{n}
\end{bmatrix}$$
Now one can combine the $2$ different kinds of dot symbols to form a generally sized matrix,
\begin{lstlisting}[language=tex]
$$\begin{bmatrix}
v_{11} & \dots & v_{1j} \\
\vdots & & \vdots \\
v_{i1} & \dots & v_{ij}
\end{bmatrix}$$
\end{lstlisting}
The \texttt{LaTex} code above, produces the output shown below,
$$\begin{bmatrix}
v_{11} & \dots & v_{1j} \\
\vdots & & \vdots \\
v_{i1} & \dots & v_{ij}
\end{bmatrix}$$
The center element of the matrix above is left blank, but one can populate it with \texttt{\textbackslash ddots} to invoke diagonal dots. That would make the matrix notation look more complete,
\begin{lstlisting}[language=tex]
$$\begin{bmatrix}
v_{11} & \dots & v_{1j} \\
\vdots & \ddots & \vdots \\
v_{i1} & \dots & v_{ij}
\end{bmatrix}$$
\end{lstlisting}
The \texttt{LaTex} code above, produces the output shown below,
$$\begin{bmatrix}
v_{11} & \dots & v_{1j} \\
\vdots & \ddots & \vdots \\
v_{i1} & \dots & v_{ij}
\end{bmatrix}$$
From all of this, the following example should be readable by now,
\begin{lstlisting}[language=tex]
$$\begin{bmatrix}
\partial x_{1} & \partial x_{2} & \dots & \partial x_{n}
\end{bmatrix}^{T} = \begin{bmatrix}
\partial x_{1} \\ \partial x_{2} \\ \vdots \\ \partial x_{n}
\end{bmatrix}$$
\end{lstlisting}
The \texttt{LaTex} code above, produces the output shown below,
$$\begin{bmatrix}
\partial x_{1} & \partial x_{2} & \dots & \partial x_{n}
\end{bmatrix}^{T} = \begin{bmatrix}
\partial x_{1} \\ \partial x_{2} \\ \vdots \\ \partial x_{n}
\end{bmatrix}$$

%Seperator
%Seperator
%Seperator
%Seperator
%Seperator
\section{Document Organization}
\begin{comment}
\begin{lstlisting}[language=tex]
\end{lstlisting}
The \texttt{LaTex} code above, produces the output shown below,
\end{comment}
If a document gets sufficiently large, putting all the content together without organization confuses the reader. There are a few important commands for text writing. If $2$ lines of text is separated by more than a single line of space in between, \texttt{LaTex} would interpret that as a command to put a newline character in between the lines of text. Alternatively, one can invoke the \texttt{\textbackslash \textbackslash} command.
\\~\\If one wants more spacing between the lines, one can invoke the newline command successively like this \texttt{\textbackslash \textbackslash \texttildelow \textbackslash \textbackslash}. The author can chain the newline command more more than just $2$ times, like this: \texttt{\textbackslash \textbackslash \texttildelow \textbackslash \textbackslash \texttildelow \textbackslash \textbackslash } 

%Seperator
%Seperator
%Seperator
%Seperator
\subsection{Organizational Hierarchy}
\begin{comment}
\end{comment}
The \texttt{LaTex} typesetting system recognizes $7$ levels of organizational hierarchy, those are listed below from the most general (large) to the most specific (small):
\begin{enumerate}
\item \texttt{part}: To use, invoke command \texttt{\textbackslash part\{Enter Name of Part\}} 
\item \texttt{chapter}: To use, invoke command \texttt{\textbackslash chapter\{Enter Name of chapter\}} 
\item \texttt{section}: To use, invoke command \texttt{\textbackslash section\{Enter Name of Section\}} 
\item \texttt{subsection}: To use, invoke command \texttt{\textbackslash subsection\{Enter Name of Subsection\}} 
\item \texttt{subsubsection}: To use, invoke command \texttt{\textbackslash subsubsection\{Enter Name of subsubsection\}}
\item \texttt{paragraph}: To use, invoke command \texttt{\textbackslash paragraph\{Enter Name of paragraph\}}
\item \texttt{subparagraph}: To use, invoke command \texttt{\textbackslash subparagraph\{Enter Name of subparagraph\}}
\end{enumerate}
Below shows an example of a document that contains all $7$ organizational structures. You are encouraged to compile the document for yourself to see the true appearance of the document.
\lstinputlisting[language=tex]{Organization.tex}
Personally when writing documents, I like to begin from the \texttt{section} level and expand the organizational tree to be more general or more specific on an as-needed basis. Usually, I would stop at the \texttt{subsubsection} level at the most specific and would stop at the \texttt{part} level for large documents. For Smaller documents, I would stop at the \texttt{section} level as the most general organizational structure.

%Seperator
%Seperator
%Seperator
%Seperator
\subsection{Adding Titles}
\begin{comment}
\end{comment}
The Title commands are added after the \texttt{\textbackslash begin\{document\}} command. Below shows an example of how to make titles in a new document:
\lstinputlisting[language=tex]{Titles.tex}
Let us analyze the commands. The \texttt{\textbackslash title} command specifies the main message of the title. The \texttt{\textbackslash author} and \texttt{\textbackslash date} commands are self explanatory. The \texttt{\textbackslash maketitle} command tells the compiler to truly make the title. If this command is absent, the title would not be printed. This gives the author an option of "showing" or "hiding" the title of the document. To "show" the title of the document, the author just has to invoke the \texttt{\textbackslash maketitle} command. To "hide" the title, just comment the \texttt{\textbackslash maketitle} command out.
\\~\\Try commenting out the various parts of the title command to see which parts of the title the commands correspond to.

%Seperator
%Seperator
%Seperator
%Seperator
\subsection{Table of Contents}
\begin{comment}
\end{comment}
Table of content generation in \texttt{LaTex} is incredibly easy. Just invoke the command \texttt{\textbackslash tableofcontents} wherever you wish to write the table of contents. Below shows an example of how to use the \texttt{\textbackslash tableofcontents} command from earlier,
\lstinputlisting[language=tex]{Table_Of_Content.tex}

%Seperator
%Seperator
%Seperator
%Seperator
%Seperator
\section{Miscellaneous}
\begin{comment}
\end{comment}

%Seperator
%Seperator
%Seperator
%Seperator
\subsection{Image Insertion}
\begin{comment}
\end{comment}
Image insertion can be done using the \texttt{graphicx} package. To load the package, use the following command,
\begin{lstlisting}[language=tex]
\usepackage{graphicx}
\end{lstlisting}
The rest of the exmaples here assumes that an image named \texttt{image.png} is located on the same directory as the \texttt{.tex} file. The example below is a simple way to include images,
\begin{lstlisting}
\\~\\\includegraphics[scale=0.3]{image.png}
\end{lstlisting}
The size of the image that is printed can be rescaled by changing the \texttt{scale} from $0.3$ to say $0.4$. One is free to change the scaling to make the image size display to one's preference. For most documents, this should be sufficient, but sometimes we need to show documents with captions and perhaps refer to it later. The example below shows how to do this,
\begin{lstlisting}[language=tex]
\begin{figure}[H]\centering
\includegraphics[scale=0.3]{image.png}
\caption{This is an image}
\label{image labelling}
\end{figure}
\end{lstlisting}
The example above requires the \texttt{float} package though. That package can be loaded like any other package with the \texttt{\textbackslash usepackage} command. Below is the command to load the \texttt{float} package:
\begin{lstlisting}[language=tex]
\usepackage{float}
\end{lstlisting}

%Seperator
%Seperator
%Seperator
%Seperator
\subsection{Programming Insertion}
\begin{comment}
\end{comment}
To insert programs, one can use the \texttt{listings} package. To load the \texttt{listings} package,
\begin{lstlisting}[language=tex]
\usepackage{listings}
\end{lstlisting}
To include direct computer code, we can start a \texttt{listings} environment. Here is an example of including a program inside a \texttt{LaTex} document,
\lstinputlisting[language=tex]{Listings.tex}
The resulting listing however looks quite ugly. So there are a few parameters we can change inside the \texttt{listings} package that can make the listings look nicer. Here is an improved version:
\lstinputlisting[language=tex]{Improved_Listings.tex}
It is important to note that the example above uses a new package, \texttt{xcolor}. \texttt{xcolor} is used to describe colors such as \texttt{codegreen} and \texttt{codegray} and so on.

%Seperator
%Seperator
%Seperator
%Seperator
%Seperator
\section{Additional References}
\begin{comment}
\end{comment}
\texttt{Overleaf} is a fantastic rescource containing very beginner-friendly explanations on how to write in \texttt{LaTex}. \texttt{Overleaf} also contains a web-based application that allows \texttt{LaTex} document to be written through the web-browser. The link for \texttt{Overleaf}:
\\~\\\url{https://www.overleaf.com/learn/latex/Learn_LaTeX_in_30_minutes} 
\\~\\The official \texttt{LaTex} documentation can be found in the link below:
\\~\\\url{https://www.latex-project.org/help/documentation/} 
\\~\\There are many \texttt{LaTex} packages out there. The documentation for many of these packages can be found in \texttt{CTAN} (Comprehensive Tex Archive Network). The link to that is shown below:
\\~\\\url{https://ctan.org/?lang=en} 
\\~\\These are my personal preference for packages and their uses:
\begin{enumerate}
\item \texttt{amsmath}: Various mathematical utilities
\item \texttt{comment}: Allows the usage of comment blocks
\item \texttt{amssymb}: For more mathematical symbols
\item \texttt{esint}: For notations of double and triple integrals
\item \texttt{geometry}: To trim the margin of the document and also declare the layout, like landscape or portrait
\item \texttt{graphicx}: To include images in a \texttt{LaTex} document
\item \texttt{hyperref}: Generate hyperlinks on the \texttt{LaTex} document to online sites.
\item \texttt{listings}: A way to include computer programming languages in a \texttt{LaTex} document seamlessly
\item \texttt{xcolor}: A way to incorporate colors to \texttt{LaTex}. Usually used for syntax highlighting with \texttt{listings}
\item \texttt{array}: Extends the usage of tables. Table column sizes can be set through this package
\item \texttt{multicol}: Allows the document to be written in columns
\item \texttt{circuitikz}: Allows the drawing of circuit diagrams
\item \texttt{tikz}: A general drawing tool for \texttt{LaTex} 
\end{enumerate}

%Seperator
%Seperator
%Seperator
%Seperator
%Seperator
\end{center}
\end{document}
